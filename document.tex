%%This is a very basic article template.
%%There is just one section and two subsections.
\documentclass{article}
\usepackage{ucs}
\usepackage[utf8]{inputenc}
\usepackage[english, russian]{babel}
\title{Использование GPIO на Cubieboard.}
\date{}
\author{}

\begin{document}
\maketitle


\section{Что такое GPIO}

GPIO(General-purpose input/output) - это ряд специальных контактов общего
назначения, пользователь может определить для каждого из них является ли он
входом, либо выходом. 
\newline
По умолчанию назначения не определены.
\newline
Возможности GPIO:\cite{wiki:gpio}
\begin{enumerate}
  \item{может быть сконфигурировани для ввода или вывода}
  \item{GPIO контакты могут быть отключены}
  \item{на выходе может быть или 1 или 0}
  \item{можно изменять выходные значения}
  \item{часто можно использовать систему прерываний для GPIO} 
\end{enumerate}


 
\section{Особенности GPIO в Cubieboard}


\section{Использование GPIO в ОС Linux}

More plain text.

\section{Использование прерываний GPIO}

\bibliographystyle{utf8gost705u}
\bibliography{biblio}

\end{document} 
